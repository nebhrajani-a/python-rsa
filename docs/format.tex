\documentclass[11pt]{article}
%-------------------------------------------------------------------------------------------------%
\usepackage[utf8]{inputenc}
\usepackage[style = geschichtsfrkl]{biblatex}
\usepackage[a4paper, total={6in, 9in}]{geometry}
\usepackage[T1]{fontenc}
\usepackage[nottoc,numbib]{tocbibind}
\usepackage{multicol}
\usepackage{tabularx}
\usepackage{graphicx}
\usepackage{titling}
\usepackage{tocloft}
\usepackage{caption}
\usepackage{subcaption}
\usepackage{hologo}
\usepackage[hidelinks]{hyperref}
%-------------------------------------------------------------------------------------------------%
\graphicspath{ {./images/} }
\addbibresource{final_project.bib}
%-------------------------------------------------------------------------------------------------%
\renewcommand\cftsecpagefont{\normalfont}
\renewcommand{\cftsecleader}{\cftdotfill{\cftsecdotsep}}
\renewcommand\cftsecdotsep{\cftdot}
\renewcommand\cftsubsecdotsep{\cftdot}
\pretitle{
  \begin{center}
    \Huge\bfseries }
\renewcommand\maketitlehookb{
  \begin{center}
    \Large {An Investigatory Project Report}\\[0.25in] \includegraphics[scale =
      0.75]{tos_logo.jpeg}\\[0.25in] \bfseries {Submitted by}
  \end{center}
}
\predate{}
\postdate{}
\renewcommand\maketitlehookd{
  \begin{center}
    \large{ The Orchid School\\ Baner Pune 411045\\[0.3cm] \textbf{2020-2021}\\ }
  \end{center}
}
%-------------------------------------------------------------------------------------------------%
\title{Determination of Caffeine Content in Tea Samples}
\author{ Nebhrajani, Aditya V.}%\\ Magapu,Abhimanyu\\ Sharma, Galav\\ }
\date{}
%-------------------------------------------------------------------------------------------------%

\begin{document}

\maketitle
\thispagestyle{empty}

\newpage


{\center\includegraphics[scale = 0.5]{tos_logo.jpeg}\\[0.25in]} {\center\LARGE The Orchid School\\}
{\center\huge\textbf {Certificate}\\[0.5cm]}

This is to certify that Aditya V. Nebhrajani of Class XII I1 has successfully completed the Chemistry
Investigatory Project in the partial fulfillment of curriculum of Central Board of Secondary
Education (CBSE) leading to award of Annual Examination of the year 2020-2021.\\[1cm]


\begin{center}
\begin{tabularx}{\textwidth}{>{\centering\arraybackslash}X  >{\centering\arraybackslash}X}
External Examiner & Teacher In-Charge \\ & (Pooja Sharma)\\[1.5cm] Unit Head & Principal \\ (Netre
Kulkarni) & (Namrata Majhail) \\
\end{tabularx}
\end{center}

\newpage

\section*{Acknowledgements}
\addcontentsline{toc}{section}{Acknowledgements}

No project can ever be completed without the graceful contribution of many, many people.\\

I would, first and foremost, like to express my gratitude to our chemistry teacher, Ms. Pooja
Sharma for her valuable advice and inputs which enhanced the quality of my work. This project could
not have been completed successfully without her help. I would also like to thank Netre Di for her
encouragement and support, and our respected principal Namrata Di for giving me an opportunity to
work on such an intriguing topic. I would also like to acknowledge our lab assistant (?) Di for
her willingness in providing the necessary lab equipment.\\

This entire project was written on open source software, using open source programming and markup
languages. I am hence deeply indebted to the open source community at large. I would like to
separately acknowledge the Emacs\cite{emacs}, \LaTeX\ \cite{latex}, \hologo{BibTeX} \cite{bibtex}, and the
\TeX\ StackExchange community\cite{texSE} for their powerful and brilliant tools.\\

Lastly, I would like to thank our parents for their love and guidance without which nothing can be
accomplished.

\newpage
\tableofcontents

\newpage
\listoffigures
\listoftables
???


\newpage
\section{Introduction}

\subsection{Tea}
Tea is the second most widely consumed beverage in the world, second
only to water \cite{teapop}. The origin story of tea is an interesting
one, a mix of myth and fact and colored by ancient concepts of
spirituality and philosophy. \cite{teastory}$^,$\cite{teawiley}\\
According to Chinese legend, the history of tea began in 2737 B.C.E. when the Emperor Shen Nong, a
skilled ruler and scientist, accidentally discovered tea. While boiling water in the garden, a leaf
from an overhanging wild tea tree drifted into his pot. The Emperor enjoyed drinking the infused
water so much that he was compelled to research the plant further. Legend has it that the Emperor
discovered tea's medicinal properties during his research. \\
Indian history attributes the discovery of tea to Prince Bodhi-Dharma, an Indian saint who founded
the Zen school of Buddhism. In the year 520 A.D, he left India to preach Buddhism in China. To prove
some Zen principles, he vowed to meditate for nine years without sleep. It is said that towards the
end of his meditation, he fell asleep. Upon awaking, he was so distraught that he cut off his
eyelids, and threw them to the ground. Legend has it that a tea plant sprung up on the spot to
sanctify his sacrifice. \cite{teastory} \\
Tea is also an important part of history, from the East India Company to the Boston Tea
Party. \cite{teawiley} \\
Like Europe, tea initially came to America in the mid-1600s by way of the Dutch settlement of New
Amsterdam. The colony was captured by England in 1664 and renamed New York, where the tea trade
flourished amongst colonial women and wealthy colonists.
At the same time, the British East India Company had persuaded the English Parliament to implement
heavy taxes on tea by way of the Tea Act, to bolster up their failing financial position. This
allowed them to ship tea duty-fee directly to the colonists and profit by excluding the colonial
merchants. \cite{eicboston}\\
This created a tense political environment in America, resulting in dissent and the popularization
of the notion of `no taxation without representation' amongst the colonies. Political tensions came
to a climax with the Boston Tea Party, as colonists protested England's high taxes by dressing as
Native Americans and dumping tea into the water off The Company's trading ships.
The American Revolution that followed in 1776 may have set The Company back, but it still managed to
survive due to its immense size. However, Richard Twining and thousands of independent tea merchants
organized a campaign to reveal the Company's corrupt practices and pressured the English government
to end their monopoly. This caused The Company to eventually crumble in 1874. \cite{eicboston} \\
All in all, tea is a vital part of human culture, especially Eastern. Naturally, as the world became
more and more interconnected via trade routes, colonisation, and now, the Internet, tea has become a
way of connecting people of all walks of life, be it a strong black Assam tea or the light green
Japanese Sencha. Because of this variety, many tea-tasting traditions came up as well. Japanese tea
traditions are also very elegant, sometimes taking hours to drink a simple cup of tea. This has
observed psychological benefits, and is considered an art form. \cite{ceremony} \\
Recently, engineers tested an electronic nose and tongue by tea-tasting as
well. \cite{electronicnose} \\
This only reiterates how aware we must be of the tea we drink. This paper finds its
context and relevance here. I attempt to analyse the caffeine content of some popular teas using a
non-laborious method.\\

\subsection{Caffeine}
Caffeine is the world's most widely consumed psychoactive drug. Its use is legal and unregulated in
most parts of the world. Its effects on human health are mixed \cite{healtheffects} and is primarily
used as a natural stimulant. Caffeine blocks the effect of adenosine, which is a neurotransmitter
that relaxes the brain. Caffeine helps stay awake by connecting to adenosine receptors in the brain
without activating them. This blocks the effects of adenosine, leading to a reduction in percieved
tiredness. \\
Caffeine is found primarily in the tea, coffee, and cocoa plants. It's estimated \cite{80} that 80\%
of the world's population consumes caffeine on a daily basis. This number is higher for North
America, equal to almost 90\% of the population. A caffeine intake of 200 mg per dose, and up to 400
mg per day, is generally considered safe. \\
Caffeine consumption and use, however has evolved from its humble origins in tea and
coffee. Caffeine is increasingly being added to soft drinks (Coca-Cola), and foods. This has led to
deeper research into its health effects, which are of interest to us. \\
For years, caffeine has been the “drug” of choice in many cultures. Caffeine has been considered
socially acceptable because it is found in drinks like tea or coffee. People who consume a lot of
caffeine-based drinks may think they are addicted -- depending heavily on the substance. But if they
stop using such drinks, they will experience only mild symptoms of withdrawal for a few days.
The real addiction may be emotional. Many people claim they cannot start their day unless they get
their “fix,” which is, in many cases, a cup or two of coffee or tea. Some people drink tea throughout
the day. Even young people who may not touch tea or coffee are still putting caffeine into their
bodies when they have energy drinks, which have high amounts of caffeine.\\
Many studies have been done on the addictiveness of caffeine, and most conclude that caffeine
addicts people in a way not dissimilar to drugs such as nicotine and cocaine. However, it is also
linked to reducing oral and throat cancers. As with all natural substances, the dose makes the
poison. Caffeine intake for each person should be in moderation. \cite{caffhealth}$^,$
\cite{caffhealth2}$^,$ \cite{caffhealth3}\\
The beautiful cultures caffeine creates are central to the human experience: evidenced by phrases
such as `meeting over coffee', `drinking ch'a (chai or tea)', and the popularity of multinational
chains such as Starbucks. \\


\newpage

\section{Aim}

\subsection{Formal Aim}
The primary aim of this paper is to determine the caffeine content in some tea samples. The
secondary aim of this paper is to determine the acid content in the tea leaves as well. I attempt
to find a relation between the caffeine content in a tea and its acidity.

\subsection{General Objectives}
The conception of this paper has fulfilled far more than the formally stated aims. Some of these
greater general objectives are stated below. \\
The creation of this paper allowed the author to gain a deeper understanding of the
procedures of laboratory filtration and analysis. Using this data to then extrapolate the presence
and quantity of caffeine in various teas also taught the author rigourous application of the
scientific method. The author was gracefully allowed by The Orchid School and the teachers to use
the laboratory, which greatly augmented this paper and its author. \\
The actual writing of the paper after performing the experiment gave the
author insight into the correct use of technical language and tools used by professional scientists
and engineers. \\
In toto, the creation of this paper and the performance of the experiment(s) related to it greatly
enhanced the author's scientific understanding and scope of knowledge.


\newpage

\section{Theory}

\subsection{Caffeine}

Caffeine is an odourless, white-coloured (needles or powder), bitter purine which acts as a CNS (central nervous system)
stimulant. Its IUPAC name is \textbf{    1,3,7-Trimethylpurine-2,6-dione} (common name: Guaranine
Methyltheobromine), and the molecular formula is
$\mathrm{C_8H_{10}N_4O_2}$. \cite{struct1}$^,$\cite{struct2}$^,$\cite{struct3} \\

%% \begin{figure}[h]
%% \centering
%% \includegraphics[scale = 0.1]{struct.png}\\
%% \caption{Molecular structure of caffeine}
%% \end{figure}

\begin{figure}[h]
\centering
\begin{subfigure}{.4\textwidth}
  \centering
  \includegraphics[width=.4\linewidth]{struct.png}
  \caption{Molecular structure\cite{sfig1}}
  \label{struct1}
\end{subfigure}
\hspace{.14\textwidth}
\begin{subfigure}{.4\textwidth}
  \centering
  \includegraphics[width=.4\linewidth]{struct2.png}
  \caption{Ball-and-stick model\cite{sfig2}}
  \label{struct2}
\end{subfigure}
\caption{Structure of Caffeine}
\label{struct}
\end{figure}

\paragraph{Physical Properties}
Caffeine has 49.98\% carbon, 28.85\% nitrogen, 16.48\% oxygen and  5.19\% hydrogen. Caffeine’s molar
mass is about $194.19$ g/mol. It has a melting point of $\mathrm{235^{\circ}C}$ to
$\mathrm{238^{\circ}C}$. The boiling point of caffeine is $\mathrm{178^{\circ}C}$ and it has a
density of $1.23$ $\mathrm{g/cm^3}$. With a pH of 6.9, it is slightly basic.

\paragraph{Organic Classification}
Caffeine does not contain any stereogenic centers and hence is classified as an achiral
molecule. The xanthine core of caffeine contains two fused rings, pyrimidinedione\cite{chem1} and
imidazole.\cite{chem2} The pyrimidinedione in turn contains two amide functional groups that exist
predominately in a zwitterionic\cite{chem3} resonance with the location from which the nitrogen atoms are double
bonded to their adjacent amide carbons atoms. Hence, all six of the atoms within the pyrimidinedione
ring system are $\mathrm{sp^2}$ hybridized and planar. Therefore, the fused 5,6 ring core of
caffeine contains a total of ten $\pi$ electrons and hence according to Hückel's rule is aromatic.

\subsubsection{Uses of Caffeine}
\paragraph{Performance enhancement \cite{effects1}}
Caffeine reduces fatigue and drowsiness. At normal doses, caffeine has variable effects on learning
and memory, but it generally improves reaction time, wakefulness, concentration, and motor
coordination. The amount of caffeine needed to produce these effects varies between individuals.\\
Caffeine improves athletic performance in aerobic (especially endurance sports) and anaerobic
conditions. Moderate doses of caffeine (around 5 mg/kg) can improve sprint performance, cycling and
running time trial performance, endurance (i.e., it delays the onset of muscle fatigue and central
fatigue), and cycling power output. Caffeine increases the metabolic rate in adults. \cite{caffwiki}
\paragraph{Medical}
Caffeine is used as a medicine in combination with other drugs. It is sometimes used to help
premature babies to breathe. The short-term risk of this treatment seems to be that the babies
treated gain less weight than usual.\\
Caffeine is sometimes given to people after a lumbar puncture. This is used a test to see if someone has
meningitis. s used in many over the counter medicines, such as Excedrin, Midol and Anacin. The
caffeine in a cup of coffee (100–130 mg) improved pain relief when combined with paracetamol or
ibuprofen in 5–10\% of people. When combined with other analgesics, caffeine can help to alleviate
headaches and cramps. \cite{caffwiki}
\subsubsection{Effects of Caffeine}
\paragraph{Primary Short-Term Effects \cite{effects}} Caffeine is consumed by humans in coffee, tea, and cocoa
products. Within 5-30 minutes of consuming caffeine, the following effects appear.
\begin{itemize}
  \item Increased alertness and activity
  \item Restlessness, excitability and dizziness
  \item Anxiety and irritability
  \item Dehydration and dysuria
  \item Elevation in body temperature
  \item Faster breathing
  \item Tachycardia (increased heart rate)
  \item Headache and lack of concentration
  \item Stomach pains
\end{itemize}
\paragraph{Adverse Long-Term Effects \cite{effects}} Caffeine's widespread use has resulted in health concerns, primarily
because of its addictive nature. Some of these are listed below.
\begin{enumerate}
  \item{Physical}
    \begin{itemize}
      \item Coffee and caffeine can affect gastrointestinal motility and gastric acid secretion.
      \item Caffeine in low doses may cause weak bronchodilation for up to four hours in asthmatics.
      \item Doses of caffeine equivalent to the amount normally found in standard servings of tea,
        coffee and carbonated soft drinks appear to have no diuretic action.
      \item  However, acute ingestion of caffeine in large doses (at least 250–300 mg, equivalent to
        the amount found in 2–3 cups of coffee or 5–8 cups of tea) results in a short-term
        stimulation of urine output in individuals who have been deprived of caffeine for a period
        of days or weeks.\cite{Maughan2003CaffeineIA}
    \end{itemize}
  \item{Psychological}
    \begin{itemize}
      \item Minor undesired symptoms from caffeine ingestion not sufficiently severe to warrant a
        psychiatric diagnosis are common and include mild anxiety, jitteriness, insomnia, increased
        sleep latency, and reduced coordination.
      \item Minor undesired symptoms from caffeine ingestion not sufficiently severe to warrant a
        psychiatric diagnosis are common and include mild anxiety, jitteriness, insomnia, increased
        sleep latency, and reduced coordination. Caffeine can have negative effects on
        anxiety disorders. According to a 2011 literature review,
        caffeine use is positively associated with anxiety and panic disorders.\cite{Vilarim2011CaffeineCT}
    \end{itemize}
  \item{Addiction}
    \begin{itemize}
      \item  Some diagnostic models, such as the ICDM-9 and ICD-10, include a classification of
      caffeine addiction under a broader diagnostic model. Some state that certain users can become
      addicted and therefore unable to decrease use even though they know there are negative health
      effects.
    \end{itemize}
\end{enumerate}
\subsection{Determination of Caffeine and Acid Content in Tea}
\paragraph {Caffeine} Caffeine is present in tea leaves up to
4% and can be extracted by first boiling the
tea leaves with water which dissolves many
glycoside compounds in addition to
caffeine. The clear solution is then treated
with lead acetate to precipitate the
glycoside compounds in the form of lead
complex. Caffeine is then extracted from the clear filtrate by chloroform treatment. \cite{man}$^,$\cite{man2}

\paragraph{Acid}
The attempt is to find the general acidic content of tea and not specifically tannic acid. For this
reason, we do not separately extract tannic acid using $\mathrm{CaCO_3}$. In fact, most teas \emph{do
  not} contain tannic acid. Tea contains tannins, a class of astringent, polyphenolic
biomolecules of which tannic acid is a member. Some teas do not contain tannic acid, such as green
tea and black tea. Many scholarly articles and textbooks use tannic acid and tannins as
interchangeable terms, which is incorrect. As our interest is not in the nature of tannins in the
tea but instead in the relation between the general acidic content and caffeine content of the tea,
we skip over this confusion and simply titrate a heated extract of tea leaves and water against an
N/50 NaOH solution. \cite{man}$^,$\cite{post}$^,$\cite{Martin1983}


\newpage

\section{Experiment}

\subsection{Determination of Caffeine Content}
\subsubsection{Materials Required}

\paragraph{Apparatus}
Beakers, pipettes, burner, separating funnel, filter paper, weight box, analytical balance, spatula, funnel.
\paragraph{Chemicals}
Tea samples, lead acetate, chloroform, water.

\subsubsection{Procedure}
\begin{enumerate}
\item 50 grams of tea leaves were taken as sample and 150 ml of water was added to it in a
beaker.
\item  The beaker was then heated up to extreme boiling.
\item The solution was filtered and lead acetate was added to the filtrate, leading to the formation
  of a curdy brown coloured precipitate.
\item Lead acetate was added until no more precipitate could be formed. The solution was again filtered.
\item The filtrate so obtained was heated until the quantity left was 50 ml.
\item Then the solution left was allowed to cool.
\item After cooling, 20 ml of chloroform was added to it.
\item Soon after, two layers appeared in the separating funnel. The residue left behind was caffeine.
\item Then, weighed it and recorded the observations.
\item The same procedure was performed with different samples of tea leaves.
\end{enumerate}
\subsubsection{Observations}

\subsection{Determination of Acid Content}
\subsubsection{Materials Required}
\paragraph{Apparatus}
Beaker, pipette, burette stand, conical flask, burner, measuring flask, separating funnel, filter
paper, weight box, analytical balance.
\paragraph{Chemicals}
Tea samples, water, NaOH, phenolphthalein indicator.
\subsubsection{Procedure}
\begin{enumerate}
\item 10 gm of tea leaves were mixed in a beaker containing 200 ml of water.
\item The contents of the beaker were then heated constantly for about 30 minutes and the extract
  filtered out.
\item 5 ml of tea extract was taken in a conical flask and added to 20 ml of distilled water.
\item It was now shaken to prepare a homogeneous mixture and titrated against N/50 NaOH solution.
\item The same procedure was carried for the other samples of tea leaves.
\end{enumerate}
\subsubsection{Observations}
\paragraph{Calculations}

\newpage

\section{Results}
From the experiment(s), we conclude the following.
\begin{enumerate}
\item The amount of caffeine present in the tea samples is inversely proportional to the strength of the
  acids in the tea extract.
\item Different tastes of different teas is due to the variation of amount of caffeine and acids
  present.
\end{enumerate}


\newpage

\section{Author's Note}

This paper is written as a submission to the Central Board of Secondary Education (India) as a
Chemistry Investigatory Project for the year 2020-21. \\
It was noticed by the author of this paper that generally, investigatory projects, which are an
amazing opportunity to showcase research and academia-related talent, typically end up being
plagiarised experiments from textbooks or the World Wide Web. In fact, the top ten results of
an Internet search for this very project: `determination of caffeine content in tea samples', are all
nearly identical papers, with the same content and errors. I hence perceive that on a general
basis, students' participation is lacking in investigatory projects.\\
This paper is an attempt to fully utilize the opportunity gracefully granted to me by CBSE and The
Orchid School, and to raise the bar for all students. It is my hope that this will inspire new and
fruitful research at the secondary level of education.\\
It has been my attempt to write this paper in the same way as a scholarly article insofar as I
could. If there are any errors or discrepancies, they remain my own.\\
I thank you for taking the time to go through this paper, and I hope you enjoyed reading it as much
as I enjoyed writing it.\\

\newpage

\renewcommand\refname{Bibliography}
\printbibliography
\addcontentsline{toc}{section}{Bibliography}
\end{document}
